\chapter*{Úvod}\addcontentsline{toc}{chapter}{Úvod}\markboth{Úvod}{Úvod}
%---------------------------------------------------------------
\setcounter{page}{1}

Jako pedagog na střední škole si denně uvědomuji, jak náročné je vytvářet výukové materiály, které nejen předávají informace, ale také aktivně zapojují studenty a podporují jejich hlubší pochopení látky. 
Digitální technologie dnes hrají klíčovou roli ve vzdělávání, avšak jejich efektivní integrace do výukového procesu není samozřejmostí. 
Ačkoli existuje mnoho nástrojů pro tvorbu výukových materiálů, často narážím na jejich omezené možnosti přizpůsobení nebo nedostatečnou podporu interaktivity, která je pro moderní vzdělávání stále důležitější. 
Tyto překážky komplikují nejen práci učitelů, ale i~samotné studium, protože studenti potřebují materiály, které odpovídají dynamice současného vzdělávacího procesu a reflektují jejich individuální potřeby.

Cílem této práce je návrh a implementace platformy pro tvorbu a sdílení interaktivních vzdělávacích materiálů, která nabídne pedagogům flexibilní nástroj pro efektivnější přípravu výuky a studentům umožní přístup k~materiálům, jež lépe podporují jejich aktivní učení. 
Klíčovým aspektem platformy bude intuitivní prostředí, které uživatelům poskytne možnost vytvářet a upravovat obsah podle vlastních potřeb, a to s~důrazem na přehlednost, snadnou použitelnost a interaktivní prvky, které obohatí vzdělávací proces.

Práce se zaměří na analýzu existujících nástrojů, identifikaci jejich silných a slabých stránek a návrh řešení, které tyto nedostatky překoná. 
Důležitým prvkem bude také možnost komunitního zapojení, jež umožní platformě dlouhodobý vývoj a rozšiřitelnost. 
Implementovaná platforma bude testována v~reálném prostředí s~pedagogy i~studenty, aby byla ověřena její efektivita a uživatelská přívětivost. 
Závěrečná část práce se bude věnovat možnostem dalšího vývoje a potenciálním vylepšením na základě zpětné vazby od uživatelů.

Výstupy této práce by měly přispět nejen k~modernizaci výuky, ale i~k~usnadnění práce pedagogům, kteří hledají nástroje, jež jim umožní vytvářet kvalitní a dynamické vzdělávací materiály. 
Věřím, že vytvořená platforma může představovat krok směrem k~efektivnějšímu a přístupnějšímu vzdělávání, které reflektuje potřeby, jak učitelů, tak studentů.
