\chapter{Diskuze}\label{text:diskuze}

Aplikace byla implementována a otestována podle návrhu, který vznikl v analytické fázi práce.
Hlavní funkce byly realizovány včetně podpory interaktivity a možnosti komunitních rozšíření.
Platforma je v současném stavu použitelná pro běžné scénáře výuky, což potvrdilo i testování.
Z výsledků testování vyplynulo, že aplikace splňuje požadavky na použitelnost a odpovídá potřebám cílové skupiny.

Veškeré navržené části systému aktuálně považuji za smysluplné a přínosné. 
Nepředpokládám, že by bylo potřeba některou ze stávajících funkcí výrazně přepracovávat. 
Architektura i návrhové principy se v praxi osvědčily.

Přestože vývoj pokročil výrazně dál, než bylo původně plánováno, některé oblasti je nutné dále zmínit i mimo rozsah diplomové práce, aby mohly být v budoucnu zpracovány. 
V této kapitole proto shrnuji vybrané směry dalšího vývoje, které by mohly přinést výrazné zlepšení platformy:

\begin{description}
  \item[Bezpečnost rozšíření] Systém rozšíření je již implementován a aktivně používán. 
  V tuto chvíli předpokládám, že základní architektura poskytuje dostatečnou izolaci, ale nebylo provedeno hlubší testování s důrazem na odolnost proti škodlivému chování. 
  Do budoucna by bylo vhodné zaměřit se na podrobnější analýzu bezpečnostních rizik, připravit možné reakce na útoky a zvážit například zavedení lepšího mechanismu pro kontrolované spouštění nedůvěryhodného kódu včetně jednoduché změny spouštěče.

  \item[Moderování a správa obsahu] Aplikace zatím neobsahuje administraci ani nástroje pro moderování. 
  Veřejné materiály nelze mazat, rozšíření nelze odstraňovat z důvodu porušení pravidel a neexistuje možnost blokovat uživatele. 
  K tomu by bylo potřeba navrhnout a implementovat systém rolí, například na bázi RBAC, který by umožňoval řídit přístup k těmto funkcím. To je klíčové pro zajištění kvality a bezpečnosti v případě, že platformu začne využívat širší komunita.

  \item[Kolaborace a synchronizace] Základní forma kolaborace je dostupná, ale má svá omezení.
  Vylepšením by byla přesnější synchronizace bloků mezi uživateli (místo současného zamykání), možnost chatování v reálném čase a historie změn uložená na serveru. 
  To by umožnilo vracet se zpět k předchozím verzím materiálů a zlepšilo by to přehled při skupinové práci.

  \item[Rozšíření bloků] Dostupné bloky pokrývají základní scénáře, ale chybí pokročilé možnosti úprav. 
  Například textový blok neumožňuje nastavovat výšku řádku nebo pohodlně manipulovat s rozložením. 
  Do budoucna by bylo vhodné přidat další typy bloků a zjednodušit práci s existujícími.
  Užitečné by byly i bloky, které umožňují interaktivní odkrývání obsahu nebo dynamické přesouvání prvků na plátně.

  \item[Offline režim] Aplikace je zatím závislá na internetovém připojení. 
  Bylo by přínosné umožnit alespoň částečné použití v offline režimu, například prostřednictvím PWA nebo nativní aplikace. 
  Materiály by se mohly ukládat lokálně a synchronizovat později, což by rozšířilo možnosti použití mimo standardní prostředí.

  \item[Import a export] V současné době je export omezen na formáty JSON a PDF. 
  Pro větší kompatibilitu by bylo vhodné přidat podporu formátu PPTX, alespoň v základní podobě. 
  To by umožnilo jednodušší přenos materiálů do jiných prezentačních nástrojů a zvýšilo srozumitelnost pro širší publikum.

  \item[Režim prezentujícího] Prezentující v současnosti vidí stejné snímky jako sledující, pouze s několika drobnými úpravami. 
  Užitečné by bylo přidat samostatné okno s poznámkami, které by nebyly součástí prezentace. 
  Takový režim by zlepšil připravenost prezentujícího a umožnil mu lépe reagovat během výkladu.

  \item[Sdílení materiálů] Sdílení pomocí odkazu a kódu je funkční, ale zbytečně složité. 
  Do budoucna by bylo vhodné zavést možnost chráněného sdílení přes heslo nebo jednodušší autentizaci, která by umožnila například jednorázový přístup nebo skupinový přístup pomocí QR kódu.

  \item[Podpora rozšíření] Rozšíření již teď mají k dispozici řadu přístupových bodů do API, ale pro lepší neomezenou práci s materiály by bylo lepši přidat další možnosti.
  Mohlo by se jednat o lepší práci s materiály jako takovými (přístup k tvorbě snímků, úprav i mimo aktivní editor), přístup k více funkcionalitám bloků (interaktivita, animování) a obecně lepší reakce na uživatelské interakce.

  \item[Organizace materiálů] Uživatelská správa materiálů je nyní lineární, bez možnosti složek či jiné struktury. 
  To začíná být problém při větším množství dokumentů. 
  Zavedení složek a stromového zobrazení by výrazně zlepšilo přehlednost a správu obsahu pro velké množství materiálů.

  \item[Škálovatelnost a nasazení] Aktuální verze je vhodná pro malý počet uživatelů a není připravena na masové nasazení. 
  Nyní je vhodné, aby si aplikaci hostoval malý počet lidí (například škola) samostatně.
  Serverová část nedokáže řešit složitější scénáře sdílení napříč instancemi nebo více servery, jako například pro kolaboraci či sledování.
  Škálování databáze je vyřešeno pomocí MongoDB, ale aplikační logika by potřebovala adaptaci pro robustnější provoz. 
  Je nutné promyslet architekturu, která by zvládala větší zatížení a umožnila decentralizaci vlastnictví obsahu.

  \item[Šablony] Aplikace zatím neobsahuje systém šablon. 
  Ten je však zásadní pro efektivní a estetickou tvorbu prezentací na základě předdefinovaných designů. 
  Je třeba vytvořit možnost ukládání vlastních šablon, jejich sdílení mezi uživateli a přepínání při tvorbě materiálů. 
  Tím by se značně zrychlila práce, sjednotil vizuální styl výstupů a zároveň by se platforma přiblížila běžným standardům prezentačních nástrojů.

  \item[Animace] Platforma postrádá pokročilé možnosti animace prvků.
  Jediný rozumný způsob jak animovat jednotlivé části je pomocí interaktivity, a to velmi strohým způsobem.
  Pro lepší využívání platformy by bylo vhodné implementovat třeba animace skrz snímky a nebo například možnosti dělení animačních sekvencí uvnitř jednoho snímku.
  Do by dovolovalo pomocí tzv. fragmentů například postupně zobrazovat jednotlivé řádky či body v seznamech, což by vedlo k interaktivnějším materiálům.
\end{description}

Na projektu budu pracovat i nadále. 
V příštím školním roce plánuji platformu nasadit ve všech předmětech, které učím na Smíchovské střední průmyslové škole a gymnáziu.
Zároveň se pokusím zapojit další vyučující a komunitu kolem školy do vývoje rozšíření i samotné platformy.
Postupně chci vytvořit širokou sadu výukových materiálů a rozšíření, které budou sloužit jako inspirace i základ pro další tvůrce. 
Cílem je, aby kdokoliv mohl snadno vytvořit takové materiály, jaké si vymyslí.


% aplikace implementovana a otestovana dle navrhu a planů z analyzy
% vyuzitelnost platformy, probrano v testovani jiz drive
% obecne je potreba hodne na projektu dale pracovat
% napriklad cilit na rozvoj komunitnich rozsireni
% pro interaktivni prezentace je potreba vice interaktivity, prvků, wigetů
% testovani uverilo ze aplikace je pouzitelna a ja ji budu pouzivat
% prace je dokoncena, ale mohla by byt lepsi

% nize sepisuji jednotlive veci, ktere by bylo dobre do prace zahrnout (nestihli se, nebo ve vetsine byly mimo ramec diplomove prace)

% rozsireni a bezpecnost: pluginy existuji, jsem dost presvedceny ze jsou bezpecne, ale melo by se to vice otestovat, vice o tom premyslet, pripravit reakce na utoky

% moderovani: chybi administrace, pluginy a ani obsah nejde moderovat, je potreba dovolit banovani uzivatelu, mazani verejnych (zvyranznenych) materialu, mazani rozsireni, ktere neco porusuji, tim by bylo potreba vyresit role (nejspise pomoci RBAC)

% zlepseni kolaborace: kolaborace je dostatecna, ale pro lepsi experience by bylo nutne implementovat lepsi synchronizaci bloku, chatovani a treba historii materialu primo na serveru (tj. lze se vratit k verzi mesice stare napriklad)

% vice obsahu a bloku: bloky jsou velmi strohe a bylo by dobre implementovat vice typu a zlepsit pouzvani stavajich, treba moznosti do textu, ktery nedovoluje napriklad vysku radku a obecne se s nim blbe pracuje. Dobre by bylo pridat i moznost "tahani bloku" v prezentaci, napriklad na odkryvani, zobrazeni a podobne

% offline mode: materialy by melo jit v nejakem zpusobu otevirat offline v aplikaci, napriklad vytvorit nativni aplikace ci PWA, a synchronizace by byla az by byl s internetem

% moznosti importu a exportu: exportovat lze jen do JSON a PDF, bylo by lepsi mit dalsi alterantivy a obecne mit moznost z a do PPTX, coz nabizi konkurence, i kdyby to mely byt jen strohe verze dane prezentace.

% poznamky pro prezentuijiciho: mod prezentujiciho vidi jen slidy a nejake bloky jinak, nez sledujici, ale bylo by vhodne vytvorit separatni okno jen na poznamky, ktere by si mohl prezentujici cist behem prezentace

% lepsi sdileni: moznost sdileni za pomoci hesla, lepsi pristup k materialum (jen s kodem, nez url+kod)

% lepsi sprava: uzivatel ma materaily, nelze je davat do slozek a stromove struktury, to by pomohlo s prehlednosti

% nasazeni: aktualni nasazeni je velmi komorni, hodne male a neumim si predstavit ze by takto fungovalo obrovska sluzba. aplikace je ted spise delana na to, aby si ji mensi komunita lidi (skola) sama hostovala a pouzivala. do budoucna by bylo vhodne premyslet o tom, aby byla aplikace lepe skalovatelna a treba by lepe pracovala pri vice serverech ci databazich (databaze jsou vsak vyresene pomoci horizontalniho skalovani v MongoDB), server vsak neumi resit pristup napriklad sdileni materialu, kde by nebylo jasne kde je vlastne "spravce" daneho propojeni

% ja na projektu budu dale pracovat a v dalsim skolnim roce hodlam platformu pouzivat ve vsech predmetch, ktere na smichovske stredni prumyslove skole a gymnaziu budu ucit. dale se pokusim platformu sirit aby ji pouzivali dalsi vyucujici a lidi. cilem je vytvorit hodne obsahu, aby kdokoliv mohl vytvorit vetsinu veci, co si vymysli


% 

%     \item bezpecnost obecne, pluginy
%     \item mozne zlepseni: offline mode, vice kolaborace (chat, lepsi synchronziace), lepsi bloky
%     \item propagovani komunitnich rozsireni
%     \item vice importu a exportu
%     \item historie ukladana na serveru
%     \item poznamky pro prezentujiciho
% \end{itemize}