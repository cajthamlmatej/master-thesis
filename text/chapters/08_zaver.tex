\chapter*{Závěr}\addcontentsline{toc}{chapter}{Závěr}\markboth{Závěr}{Závěr}


Cílem práce bylo implementovat platformu pro tvorbu a sdílení interaktivních výukových materiálů, jako jsou prezentace, skripta či další studijní podklady. 
Dalším cílem bylo vytvořit skriptovací platformu, která umožní komunitní rozšiřování funkcionality různými způsoby.

Postup při naplňování cílů odpovídal původně navrženému vývojovému procesu. Rešeršní, analytická a návrhová část byly zpracovány sekvenčním způsobem. 
Byly analyzovány existující nástroje a aplikace pro tvorbu výukových materiálů a zároveň bylo zjišťováno, jak navrhnout takovou platformu s ohledem na bezpečnost skriptování.
Na základě analýzy byly stanoveny požadavky na výslednou aplikaci a popsán její předpokládaný způsob využití. 
Následně byla navržena architektura platformy, výběr technologií, rozdělení komponent a návrhové obrazovky.

V další fázi byla aplikace realizována pomocí iterativních metod vývoje.
Byla vytvořena webová aplikace včetně serverové části, sepsána dokumentace a připravena hlavní stránka pro prezentaci projektu.
Současně vznikla skriptovací platforma umožňující komunitní rozšíření ze strany učitelů, studentů i dalších uživatelů.
Editor a přehrávač podporují široké možnosti interaktivity a využití různých typů bloků pro zapojení studentů do výuky. 
Editor umožňuje kolaboraci více uživatelů v reálném čase, přehrávač zase zajišťuje, že studenti mohou sledovat prezentaci na svých zařízeních.
Materiály lze rovněž exportovat do dvou různých formátů pro účely zálohování. Platforma je navržena tak, aby byla dlouhodobě udržitelná a umožňovala přizpůsobení různým vyučovacím stylům i typům škol.

Výsledná aplikace prošla automatizovaným i uživatelským testováním bez zjevných problémů, které by bránily jejímu používání. 
Funkčnost byla dále ověřena i během reálné výuky, kde byly použity ukázkové materiály a otestována skriptovací rozšíření. 
Reakce testujících uživatelů -- studentů i pedagogů -- byly pozitivní, a naznačují potenciál dalšího rozšíření této platformy v praxi.

V diskuzi jsem shrnul návrhy na další rozšiřování platformy a nastínil, jak se může projekt dále vyvíjet. 
Zmínil jsem také oblasti, které považuji za vhodné pro další zaměření, včetně konkrétních úprav a rozšíření. 
% Do budoucna se nabízí i propojení s dalšími systémy nebo podpora více jazykových mutací pro mezinárodní použití.

Všechny cíle práce byly naplněny a zároveň byly realizovány i některé funkcionality nad rámec původního zadání. 
Výsledná platforma je plně funkční, nasazená na doméně a připravena k využití. 
Plánuji její další rozvoj i propagaci mezi pedagogy a širší komunitou.


% Cílem práce bylo implementovat platformu pro tvorbu a sdílení interaktivních výukových materiálů, jako jsou prezentace, skripta či další studijní podklady. 
% Dalším cílem bylo vytvořit skriptovací platformu, která umožní komunitní rozšiřování funkcionality různými způsoby.

% Postup při naplňování cílů odpovídal původně navrženému vývojovému procesu. 
% Rešeršní, analytická a návrhová část byly zpracovány sekvenčním způsobem. 
% Byly analyzovány existující nástroje a aplikace pro tvorbu výukových materiálů a zároveň bylo zjišťováno, jak navrhnout takovou platformu s ohledem na bezpečnost skriptování. 
% Na základě analýzy byly stanoveny požadavky na výslednou aplikaci a popsán její předpokládaný způsob využití. 
% Následně byla navržena architektura platformy, výběr technologií, rozdělení komponent a návrhové obrazovky.

% V další fázi byla aplikace realizována pomocí iterativních metod vývoje. 
% Byla vytvořena webová aplikace včetně serverové části, sepsána dokumentace a připravena hlavní stránka pro prezentaci projektu. 
% Současně vznikla skriptovací platforma umožňující komunitní rozšíření ze strany učitelů, studentů i dalších uživatelů. 
% Editor a přehrávač podporují široké možnosti interaktivity a využití různých typů bloků pro zapojení studentů do výuky. 
% Editor umožňuje kolaboraci více uživatelů v reálném čase, přehrávač zase zajišťuje, že studenti mohou sledovat prezentaci na svých zařízeních. 
% Materiály lze rovněž exportovat do dvou různých formátů pro účely zálohování.

% Výsledná aplikace prošla automatizovaným i uživatelským testováním bez zjevných problémů, které by bránily jejímu používání. 
% Funkčnost byla dále ověřena i během reálné výuky, kde byly použity ukázkové materiály a otestována skriptovací rozšíření.

% V diskuzi jsem shrnul návrhy na další rozšiřování platformy a nastínil, jak se může projekt dále vyvíjet. 
% Zmínil jsem také oblasti, které považuji za vhodné pro další zaměření, včetně konkrétních úprav a rozšíření.

% Všechny cíle práce byly naplněny a zároveň byly realizovány i některé funkcionality nad rámec původního zadání.
% Výsledná platforma je plně funkční, nasazená na doméně a připravena k využití. 
% Plánuji její další rozvoj i propagaci mezi pedagogy a širší komunitou.




% Cílem práce bylo implementovat platformu pro tvorbu a sdílení interaktivních výukových materiálů, jako jsou prezentace, skripta a cokoliv dalšího.
% Mimo jiné bylo cílem i implementovat skriptovací platformu pro možnosti komunitního rozšíření platformy různými způsoby.

% Postup plnění cílů práce odpovídá původnímu záměru vývojového procesu.
% Rešerše, analýza i návrh byly tvořeny sekvenčním způsobem. 
% Byly analyzovány již existující aplikace a software pro tvorbu takových materiálů, bylo zjištěno, jak co nejlépe takovou platformu z hlediska bezpečnosti skriptování navrhnout.
% Byly sepsány požadavky na výslednou aplikaci a bylo popsáno, jak se bude dále platforma používat.
% Poté byla taková platforma navržena, včetně možných technologií, způsobu rozdělení komponent a byly vytvořeny návrhové obrazovky.

% Za standardních iterativních metod vývoje byla aplikace vytvořena, tedy byla vybudována webová aplikace, server a k tomu navíc ještě byla sepsána dokumentace a realizována hlavní stránka pro propagování projektu.
% Zároveň byla vytvořena skriptovací platforma pro rozšíření aplikace komunitou vyučujících, studentů a jiných uživatelů.
% Komponenta editoru a přehrávače podporuje řadu interaktivity a možných bloků pro interaktivní zapojení do výuky.
% Editor dovoluje kolaboraci více uživatelů najednou a přehrávač podporuje možnost sledování prezentace uživateli ze svých zařízení ke skupinové interaktivitě.
% Materiály lze exportovat do dvou různých formátů pro offline zálohování.

% Výsledná aplikace je automaticky i uživatelsky otestována bez zjevných problémů, které by vadily používání platformy.
% Jednotlivě byla otestována i skriptovací platforma a ukázkové materiály byly otestovány i přímo v mé výuce.

% V diskuzi jsem uvedl, jak dále budu já (a možná i další komunita lidí) rozšiřovat tuto platformu nadále, aby její přínos byl co největší i pro jiná odvětví.
% A taktéž uvádím i to, co je dle mého další důležité zaměření platformy, včetně různých oprav a rozšíření.

% Všechny cíle práce byly splněny a byla implementována řada dalších funkcionalit i mimo záměr této diplomové práce.
% Výsledná platforma je použitelná ve výuce, je nasazena na doméně a připravena k použití kýmkoliv.
% Nadále budu aplikaci rozšiřovat a budu se snažit ji propagovat i pro použití ostatními vyučujícími a jinými lidmi.