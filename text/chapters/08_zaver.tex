\chapter*{Závěr}\addcontentsline{toc}{chapter}{Závěr}\markboth{Závěr}{Závěr}


Cílem práce bylo implementovat platformu pro tvorbu a sdílení interaktivních výukových materiálů, jako jsou prezentace, skripta či další studijní podklady. 
Dalším cílem bylo vytvořit skriptovací platformu, která umožní komunitní rozšiřování funkcionality různými způsoby.

Postup při naplňování cílů odpovídal původně navrženému vývojovému procesu. Rešeršní, analytická a návrhová část byly zpracovány sekvenčním způsobem. 
Byly analyzovány existující nástroje a aplikace pro tvorbu výukových materiálů a zároveň bylo zjišťováno, jak navrhnout takovou platformu s~ohledem na bezpečnost skriptování.
Na základě analýzy byly stanoveny požadavky na výslednou aplikaci a popsán její předpokládaný způsob využití. 
Následně byla navržena architektura platformy, výběr technologií, rozdělení komponent a návrhové obrazovky.

V další fázi byla aplikace realizována pomocí iterativních metod vývoje.
Byla vytvořena webová aplikace včetně serverové části, sepsána dokumentace a připravena hlavní stránka pro prezentaci projektu.
Současně vznikla skriptovací platforma umožňující komunitní rozšíření ze strany učitelů, studentů i~dalších uživatelů.
Editor a přehrávač nabízejí rozsáhlé možnosti interaktivity a využívání různorodých bloků, což umožňuje aktivní zapojení studentů do vzdělávacího procesu.
Editor umožňuje kolaboraci více uživatelů v~reálném čase, přehrávač zase zajišťuje, že studenti mohou sledovat prezentaci na svých zařízeních.
Materiály lze rovněž exportovat do dvou různých formátů pro účely zálohování. Platforma je navržena tak, aby byla dlouhodobě udržitelná a umožňovala přizpůsobení různým vyučovacím stylům i~typům škol.

Výsledná aplikace prošla automatizovaným i~uživatelským testováním bez zjevných problémů, které by bránily jejímu používání. 
Funkčnost byla dále ověřena i~během reálné výuky, kde byly použity ukázkové materiály a otestována skriptovací rozšíření. 
Reakce testujících uživatelů -- studentů i~pedagogů -- byly pozitivní a naznačují potenciál dalšího rozšíření této platformy v~praxi.

V diskuzi jsem uvedl několik návrhů na další rozšíření platformy a přiblížil, jakým způsobem by se projekt mohl dále rozvíjet.
Uvedl jsem rovněž oblasti, které podle mého názoru stojí za to, se na ně dále zaměřit, včetně určitých úprav a rozšíření.


Veškeré cíle této práce byly splněny a navíc byly implementovány i~některé funkce přesahující původní rozsah zadání.
Výsledná platforma je plně funkční, nasazená na doméně a připravena k~využití. 
Plánuji její další rozvoj i~propagaci mezi pedagogy a širší komunitou.
