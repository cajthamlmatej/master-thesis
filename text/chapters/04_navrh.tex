\chapter{Návrh}

\begin{chapterabstract}
Tato kapitola ...
\end{chapterabstract}

\section{Architektura}

Celá aplikace bude rozdělena do několika částí, které v této kapitole vysvětlím a popíšu svá rozhodnutí.

\subsection{Klient}\label{text:navrh/klient}

Klientem celé aplikace bude webová stránka, ve které budou uživatelé vytvářet, přehrávat a procházet materiály. 
Obecně na této části budou uživatelé ovládat celou aplikaci.
Webová stránka je vhodná, protože to dovolí rychlejší přístup ke všem funkcionalitám bez nutnosti cokoliv instalovat.
Instalace taktéž není vhodná kvůli velkému počtu používání materiálů studenty, kterým by toto stahování mohlo přijít nevhodné.
Později by šlo z aplikace vytvořit nativní aplikaci na desktop či telefony pomocí enkapsulace webové stránky.

Tato část bude pomocí REST API komunikovat se serverovou částí za pomocí protokolu HTTP, ze které bude získávat veškerá potřebná data.
Restful API je ve zkratce takové API, které pracuje se zdroji jako takovými a dovoluje nad nimi volat klasické akce jako čtení, přidání, smazání a změnu (CRUD).
Mimo jiné definuje další standardizované chování, které by toto API mělo mít.
Tento způsob dovoluje tvořit tzv. \enquote{thick klienty}, které vysvětlím a zdůvodním později.

Klient tedy poběží ve webovém prohlížeči uživatele.
Díky tomu tedy musí být klient definován a stylován za pomocí HTML a CSS.
Kontrola samotné stránky musí být ve skriptovacím jazyce JavaScript.
Pro bezpečnější a udržitelnější kód však v této aplikaci použiji typovanou nástavbu TypeScript, která se kompiluje do čistého JavaScriptu.
Klient bude tvořen v myšlence \enquote{thick klienta}, tedy většinu funkcionalit by měl implementovat klient za pomocí dat a služeb, které dostane (má k dispozici) od serveru.
Toto dovoluje odlehčit server náročnými požadavky (překreslením HTML na serveru při každé akci a podobně) a soustředit se pouze na to důležité.
Klient taktéž bude moci být používán s menší závislostí na serveru a při správném nastavení například i dále dovolovat čtení, i když uživatel nebude připojen k internetové síti.
Nevýhodou je zátěž a dostupné funkcionality na zařízení, ve kterém webová stránka poběží, ale to by na dnešních zařízeních neměl být žádný problém.

V kapitole \ref{text:vykreslovani} zmiňuji různé způsoby vykreslování obsahu na webových stránkách.
I když je analýza směřována zejména na obsah, tak podobné myšlenky, zejména z kapitoly \ref{text:vykreslovani/html} lze aplikovat i na zbytek webových stránek.
Zjednodušeně, tvoření komplexních webových aplikací s mnohými funkcionalitami (což v tomto návrhu pro aplikaci plánuji) nelze dělat udržitelně pouze s \enquote{čistými} funkcionalitami HTML, CSS a JS.
Pro komplexní \enquote{thick klient} webovou aplikaci je nutné v moderním světě webového vývoje použít framework, který za nás bude řešit nejen uvedené:

\begin{itemize}
    \item Správu stavů aplikace a synchronizaci dat mezi komponentami.
    \item Efektivní aktualizaci a vykreslování uživatelského rozhraní.
    \item Modularizaci kódu pro lepší přehlednost a údržbu.
    \item Optimalizaci výkonu při práci s DOM (například virtualizaci seznamů).
    \item Zajištění kompatibility s různými zařízeními a prohlížeči.
    \item Možnost snadného rozšíření o další funkcionality pomocí dostupných knihoven a ekosystému.
    \item Automatizaci testování a zajištění stability kódu.
    % \item Podporu pro server-side rendering (SSR) nebo static site generation (SSG), pokud je to potřeba.
\end{itemize}

Jako webový framework pro klient použiji knihovnu Vue, která je známá\todo{zdroj} pro svoji jednoduchost, stabilitu a komunitou.
Dobrou alternativou by byl například React, avšak s ním nemám tak velké zkušenosti a z výběru z těchto dvou nedostanu žádné benefity na více.
Vue podporuje programování za pomocí TypeScript.

\subsubsection{Vrstvy a zodpovědnosti}

Klientská (a i serverová) část bude dále rozdělena na další části (tzv. vrstvy –- layers) pro zjednodušení celého systému zodpovědností v aplikaci.
To povede k přehlednějšímu kódu a jednodušším změnám v celé aplikaci.
V realizované aplikaci budou vrstvy pak dále děleny a bude vytvořena další abstrakce.
Níže jsou naplánované veškeré vrstvy pro klientskou část.

\begin{description}
    \item[Komponenty] Které zajišťuji unifikované chování pro často opakující se prvky na stránce tak, aby jejich tvorba byla co nejrychlejší a dovolovala přehlednější kód. Komunikují se stránkami pomocí dvoucestných vazeb či událostí.
    \item[Stránky] Které seskupují komponenty a vlastní funkcionality do jednotlivých stránek aplikace. Předávají komponentám data a určují jim, jak se mají chovat či zobrazovat.
    \item[Perzistence] Které jsou často nazývané ve webovém inženýrství jako obchody (stores), které globálně ukládají data pro celou aplikace, aby bylo jedno místo (či místa), kde je lze najít. Slouží k tomu, aby se o data nežádalo více, než je nutné (caching). Mohou dovolovat ukládat data zcela perzistentně i mezi načtením stránky pro rychlejší prvotní načtení. Komunikují se serverem pomocí další vrstvy.
    \item[Komunikace] Což zajišťuje komunikaci se serverem pomocí Restful API serveru. Určují a kontrolují jaká data od serveru přijdou, určují lokální rozhraní pro CRUD operace s danými zdroji a mapují je na lokální objekty.
\end{description}

\subsubsection{Editor a přehrávač}

Nejklíčovější částí klienta je editor a přehrávač materiálů, které budou uživatelé tvořit.
Editor i přehrávač jsou příliš komplexní a vzhledem k cíli práce vytvořit jednoduše rozšiřitelnou aplikaci (a to i díky komunitnímu rozšíření) a proto není vhodné pro tyto části webové aplikace použít zmíněný framework.
Rozšiřování frameworku by bylo velmi zvláštní a nebylo by to příliš intuitivní.
Hlavním problémem je však ztracení jakékoliv kontroly, jak a kdy se vykresluje daný obsah, co framework spravuje.
Z tohoto důvodu musí být takto komplexní komponenta, tedy editor a přehrávač, manuálně spravována.
Naštěstí framework Vue dovoluje označit část kódu jako spravovanou někým jiným a tento prvek nebude překreslovat, pokud se nezmění jeho rodič (a jeho rodič a tak dále). 
Takovéto fundamentální rozdělení taktéž dovolí, že editor se může velmi jednoduše přidat i do jiných projektů či komponent.

V pozdější kapitole \ref{text:navrh/obsah} rozepíši, jak hodlám řešit to, jak se bude obsah zobrazovat v editoru a přehrávači, jak se bude upravovat, definovat a podobně.

\subsection{Server}\label{text:navrh/server}

Cílem serverové části je poskytovat data klientské části a to kvůli \enquote{thick klienta}. 
Tyto data bude server získávat z dalších služeb a zejména z databáze, ve které budou data perzistentně uložena.
O databázi budu rozepisovat v kapitole \ref{text:navrh/databaze}.
Mezi další služby bude patřit e-mailový server či různá další externí API třetích služeb (například na získávání obrázků a tak dále).

Poskytované API, což jsem zmínil v kapitole \ref{text:navrh/klient}, bude Restful a tedy bude pracovat jako se zdroji. 
Alternativní způsob designu API je například Simple Object Access Protocol (SOAP), který dovoluje volání akcí na serveru.
REST API je standard mezi moderními komunikacemi a vzhledem k cíli vytvořit rozšiřitelnou aplikaci je REST nejvhodnější možnost.

Server bude získávat data z URL adresy, těla a hlavičky požadavku.
Požadavky i odpovědi na a ze serveru budou komunikovat zejména v JavaScript Object Notation (JSON).
JSON je jednoduchý formát, který se jednoduše implementuje skoro v každém jazyce a je postaven na bázi seznamu klíčů a příslušných hodnot.
Hodnoty mají mnoho podporovaných datových typů.
Server by měl být však připraven na jakýkoliv formát dat.
Mělo by být možné velmi jednoduše změnit například na formát Extensible Markup Language (XML).

Serverová a klientská část by měly spolu sdílet rozhraní definující vstupní a výstupní data aplikace.
Těmto objektům se říká Data Transfer Object (DTO).
Tyto objekty dovolují jednoduše typovat data, které vstupují a vystupují do a respektive z REST API.

Některé endpointy a zdroje API musí být zabezpečené pomocí autorizace a autentizace. 
O zabezpečení budu mluvit v pozdější kapitole \ref{text:navrh/auth}.
Některé endpointy musí být limitované i rate-limitery, tedy že omezí počet volání daného endpointu pro konkrétního uživatele či globálně.
Jedná se např. o stránku s registrací.

Samotná serverová část bude tvořena v jazyce TS.
Volba TS je stejná jako v klientské části (viz kapitola \ref{text:navrh/klient}) a to tedy, že nutí programátora psát udržitelný kód, který je staticky typově kontrolován.

Implementace komunikačního uzlu je možná přes řadu způsobů a knihoven.
V JS, resp. TS ekosystému patří mezi nejoblíbenější backendový rámec knihovna Express.js (často taktéž Express).
Express je knihovna k vytvoření HTTP serveru, která dosahuje velmi úctyhodných rychlostí odpovědi a je velmi jednoduchá na pochopení.
Alternativní jako nejoblíbenější bývá\todo{zdroj} zmiňována knihovna Fastify.
Právě Express jsem se rozhodl použít díky své malé velikosti, komunitě a jeho velké stopě v ekosystému.
Express má řadu předpřipravených funkcionalit jako jsou routery, middleware a mnoho dalšího.

I přes to, že má Express mnoho připraveného, jeho používání bez jakékoliv nástavby je vhodné jen pro projekty, kde jde zejména o rychlost, a to i rychlost vývoje.
Pro udržitelnější servery je vhodné použít nějaký framework, který staví nad těmito knihovnami a dovoluje rychlejší zápis kontrolérů, modulů a služeb s pomocí Dependency Injection (DI).

Mezi nejznámější frameworky v JS ekosystému, které používají (nebo dovolují) Express, patří například NestJS či Koa.
Mezi nejvíce\todo{citace} oblíbené patří však NestJS a to zejména díky své velké komunitě, dokumentaci a rychlosti.
NestJS řeší vše dříve uvedené a mnoho dalšího.
Dovoluje rapidně tvořit udržitelné webové servery s HTTP API a podporuje řadu způsobů a knihoven návrhu.
Jedná se např. o passport, Mongo, Prisma, fronty úkolů, volání úloh, CORS, verzování API a mnoho dalšího.
Proto jsem se rozhodl NestJS s pomocí Express použít a to za pomoci TypeScriptu.

Server nejspíše bude používat i řadu dalších knihoven, například na odesílání e-mailů, jejich formátování a například na generování přihlašovacích tokenů pomocí JWT (viz kapitola \ref{text:navrh/auth}).

\subsubsection{Vrstvy a zodpovědnosti}

I serverová část bude rozdělena do vrstev. 
Základ těchto vrstev je definován rovnou ve vybraném frameworku NestJS.
Avšak vrstev tam je definováno přespříliš, a proto hodlám používat jen část z nich.
V realizované aplikaci budou vrstvy pak dále děleny a bude vytvořena další abstrakce.
Níže jsou naplánované veškeré vrstvy pro serverovou část.

\begin{description}
    \item[Moduly] které sumarizují následující vrstvy do jednotlivých balíčků, se kterými se dá jednoduše pracovat. Definují co se má exportovat a importovat v rámci DI.
    \item[Služby] které sumarizují společné funkcionality do jednoo balíčku, aby se kód nemusel opakovat. Vztahují závislosti a agregují další služby a modely. Dají se sem počítat i tzv. Guardy, které kontrolují zda bude požadavek potvrzen a v aplikaci dále předáván do funkcí kontroleru.
    \item[Kontroléry] které tvoří REST API, volají funkce služeb, určují vstupní a výstupní data včetně návratových kódů.
    \item[Modely] které definují schéma databáze, respektive jednotlivých entit uvnitř.
\end{description}


\subsection{Databáze}\label{text:navrh/databaze}

Databázová část je zodpovědná za ukládání dat a poskytování rozhraní pro čtení a úpravu daných dat. 
V této části je potřeba databáze, respektive systém řízení báze dat (SŘBD).

Mezi zvažované možnosti jsem prvně přemýšlel o PostgreSQL a MongoDB.
Tyto dvě databáze poskytují základní spektrum pro databáze a to jsou~\cite{irena2015big, marek2018sql} SQL a NoSQL databáze. 
Aplikace jako celý cíl má stanovené to, že má být v budoucnu rozšiřitelná.
Je možné si představit, že tato aplikace bude v budoucnu ukládat velmi mnoho dat a bude potřeba, aby databázový stroj podporoval horizontální i vertikální škálování.
Podobně data, která budou v databázi uložena, nemají jasně stanovenou strukturu a obecně se bude jednat o velmi komplexní objekty.
V této aplikaci je lepší ztratit konzistenci dat, které např. MongoDB negarantuje (resp. je v módu \enquote{někdy bude konzistentní}~\cite{irena2015big}), důležitá je dostupnost a odolnost k přerušení.
I když budou na sebe data ukazovat (viz datový model v kapitole \ref{text:analyza/datovymodel}), jsou tyto propojení velmi triviální a spíše je vhodné dokumenty do sebe vnořovat.
Proto nevidím důvod na použití relační databáze a to i díky struktuře dat.

Alternativou s podobnými vlastnostmi~\cite{irena2015big} jako MongoDB je CouchDB a Firebase Firestore. 
CouchDB je dokumentová databáze, která využívá JSON pro ukládání dat, podporuje replikaci a je navržena pro distribuované systémy. 
Její hlavní výhodou je tzv. \enquote{multi-master} replikace, která umožňuje synchronizaci mezi více instancemi databáze. 
Firestore, jako součást ekosystému Firebase~\cite{firebase}, nabízí jednoduché škálování a real-time synchronizaci, což je výhodné pro aplikace s požadavky na okamžitou aktualizaci dat mezi uživateli. 
Nicméně, Firestore je úzce spjat s Google Cloud platformou a může být nákladnější v závislosti na objemu provozu a požadavcích na škálovatelnost. 
Vzhledem k těmto faktorům jsem se rozhodl zvolit MongoDB, která poskytuje dostatečnou flexibilitu pro strukturu dat i škálovatelnost bez závislosti na konkrétním poskytovateli cloudových služeb.

Díky výběru knihovny Nest (viz kapitola \ref{text:navrh/server}) je vhodné pro databázi použít takový driver, aby s ním i daná knihovna uměla pracovat.
V základním rozpoložení umí Nest pracovat mimo jiné s drivery (resp. ODM/ORM) Mongoose a Prisma~\cite{nest_database}.
S Mongoose mám velmi velké zkušenosti a používám ho i v projektech mimo Nest.
Prisma je na druhou stranu velmi populární nejen pro MongoDB, ale i pro ostatní databázové stroje.
Prisma má avšak velké problémy s MongoDB -- například se jedná o pomalejší dotazování\footnote{např. issue: \url{https://github.com/prisma/prisma/issues/16916}} či často je nutné velmi složitě nastavovat server~\cite{prisma_2025}.
Definice entit a jejich závislostí je taktéž velmi omezená.
Proto jsem se rozhodl použít s MongoDB právě knihovnu ODM mongoose.

\subsection{Databázové schéma}

\section{Obsah}\label{text:navrh/obsah}

\section{Uživatelské prostředí}

\subsection{Komponenty}

\subsection{Návrhové obrazovky}

\section{Autentizace a autorizace}\label{text:navrh/auth}

\section{Návrh API}

