\chapter{Návrh}

\begin{chapterabstract}
Tato kapitola ...
\end{chapterabstract}


% \section{Požadavky a funkcionality}

\section{Architektura}

Celá aplikace bude rozdělena do několika částí, které v této kapitole vysvětlím a popíšu svá rozhodnutí.

\subsection{Klient}\label{text:navrh/klient}

Klientem celé aplikace bude webová stránka, ve které budou uživatelé vytvářet, přehrávat a procházet materiály. 
Obecně na této části budou uživatelé ovládat celou aplikaci.
Webová stránka je vhodná, protože to dovolí rychlejší přístup ke všem funkcionalitám bez nutnosti cokoliv instalovat.
Instalace taktéž není vhodná kvůli velkému počtu používání materiálů studenty, kterým by toto stahování mohlo přijít nevhodné.
Později by šlo z aplikace vytvořit nativní aplikaci na desktop či telefony pomocí enkapsulace webové stránky.

Tato část bude pomocí REST API komunikovat se serverovou částí za pomocí protokolu HTTP, ze které bude získávat veškerá potřebná data.
Restful API je ve zkratce takové API, které pracuje se zdroji jako takovými a dovoluje nad nimi volat klasické akce jako čtení, přidání, smazání a změnu (CRUD).
Mimo jiné definuje další standardizované chování, které by toto API mělo mít.
Tento způsob dovoluje tvořit tzv. \enquote{thick klienty}, které vysvětlím a zdůvodním později.

Klient tedy poběží ve webovém prohlížeči uživatele.
Díky tomu tedy musí být klient definován a stylován za pomocí HTML a CSS.
Kontrola samotné stránky musí být ve skriptovacím jazyce JavaScript.
Pro bezpečnější a udržitelnější kód však v této aplikaci použiji typovanou nástavbu TypeScript, která se kompiluje do čistého JavaScriptu.
Klient bude tvořen v myšlence \enquote{thick klienta}, tedy většinu funkcionalit by měl implementovat klient za pomocí dat a služeb, které dostane (má k dispozici) od serveru.
Toto dovoluje odlehčit server náročnými požadavky (překreslením HTML na serveru při každé akci a podobně) a soustředit se pouze na to důležité.
Klient taktéž bude moci být používán s menší závislostí na serveru a při správném nastavení například i dále dovolovat čtení, i když uživatel nebude připojen k internetové síti.
Nevýhodou je zátěž a dostupné funkcionality na zařízení, ve kterém webová stránka poběží, ale to by na dnešních zařízeních neměl být žádný problém.

V kapitole \ref{text:vykreslovani} zmiňuji různé způsoby vykreslování obsahu na webových stránkách.
I když je analýza směřována zejména na obsah, tak podobné myšlenky, zejména z kapitoly \ref{text:vykreslovani/html} lze aplikovat i na zbytek webových stránek.
Zjednodušeně, tvoření komplexních webových aplikací s mnohými funkcionalitami (což v tomto návrhu pro aplikaci plánuji) nelze dělat udržitelně pouze s \enquote{čistými} funkcionalitami HTML, CSS a JS.
Pro komplexní \enquote{thick klient} webovou aplikaci je nutné v moderním světě webového vývoje použít framework, který za nás bude řešit nejen uvedené:

\begin{itemize}
    \item Správu stavů aplikace a synchronizaci dat mezi komponentami.
    \item Efektivní aktualizaci a vykreslování uživatelského rozhraní.
    \item Modularizaci kódu pro lepší přehlednost a údržbu.
    \item Optimalizaci výkonu při práci s DOM (například virtualizaci seznamů).
    \item Zajištění kompatibility s různými zařízeními a prohlížeči.
    \item Možnost snadného rozšíření o další funkcionality pomocí dostupných knihoven a ekosystému.
    \item Automatizaci testování a zajištění stability kódu.
    % \item Podporu pro server-side rendering (SSR) nebo static site generation (SSG), pokud je to potřeba.
\end{itemize}

Jako webový framework pro klient použiji knihovnu Vue, která je známá\todo{zdroj} pro svoji jednoduchost, stabilitu a komunitou.
Dobrou alternativou by byl například React, avšak s ním nemám tak velké zkušenosti a z výběru z těchto dvou nedostanu žádné benefity na více.
Vue podporuje programování za pomocí TypeScript.

\subsubsection{Vrstvy a zodpovědnosti}

Klientská (a i serverová) část bude dále rozdělena na další části (tzv. vrstvy –- layers) pro zjednodušení celého systému zodpovědností v aplikaci.
To povede k přehlednějšímu kódu a jednodušším změnám v celé aplikaci.
V realizované aplikaci budou vrstvy pak dále děleny a bude vytvořena další abstrakce.
Níže jsou naplánované veškeré vrstvy pro klientskou část.

\begin{description}
    \item[Komponenty] které zajišťuji unifikované chování pro často opakující se prvky na stránce tak, aby jejich tvorba byla co nejrychlejší a dovolovala přehlednější kód. Komunikují se stránkami pomocí dvoucestných vazeb či událostí.
    \item[Stránky] které seskupují komponenty a vlastní funkcionality do jednotlivých stránek aplikace. Předávají komponentám data a určují jim, jak se mají chovat či zobrazovat.
    \item[Perzistence] často nazývané ve webovém inženýrství jako obchody (stores), které globálně ukládají data pro celou aplikace, aby bylo jedno místo (či místa), kde je lze najít. Slouží k tomu, aby se o data nežádalo více, než je nutné (caching). Mohou dovolovat ukládat data zcela perzistentně i mezi načtením stránky pro rychlejší prvotní načtení. Komunikují se serverem pomocí další vrstvy.
    \item[Komunikace] se serverem pomocí Restful API serveru. Určují a kontrolují jaká data od serveru přijdou, určují lokální rozhraní pro CRUD operace s danými zdroji a mapují je na lokální objekty.
\end{description}

\subsubsection{Editor a přehrávač}

Nejklíčovější částí klienta je editor a přehrávač materiálů, které budou uživatelé tvořit.
Editor i přehrávač jsou příliš komplexní a vzhledem k cíli práce vytvořit jednoduše rozšiřitelnou aplikaci (a to i díky komunitnímu rozšíření) a proto není vhodné pro tyto části webové aplikace použít zmíněný framework.
Rozšiřování frameworku by bylo velmi zvláštní a nebylo by to příliš intuitivní.
Hlavním problémem, je však ztracení jakékoliv kontroly, jak a kdy se vykresluje daný obsah, co framework spravuje.
Z tohoto důvodu musí být takto komplexní komponenta, tedy editor a přehrávač, manuálně spravována.
Naštěstí framework Vue dovoluje označit část kódu jako spravovanou někým jiným a tento prvek nebude překreslovat, pokud se nezmění jeho rodič (a jeho rodič a tak dále). 

V pozdější kapitole \ref{text:navrh/obsah} rozepíši, jak hodlám řešit to, jak se bude obsah zobrazovat v editoru a přehrávači, jak se bude upravovat, definovat a podobně.

\subsection{Server}

Cílem serverové části je poskytovat data klientské části a to kvůli \enquote{thick klienta}. 
Tyto data bude server získávat z dalších služeb a zejména databáze, ve které budou data perzistentně uložena.
O databázi budu rozepisovat v kapitole \ref{text:navrh/databaze}.
Mezi další služby bude patřit e-mailový server či různá další externí API třetích služeb (například na získávání obrázků a tak dále).

Poskytované API, což jsem zmínil v kapitole \ref{text:navrh/klient}, bude Restful a tedy bude pracovat jako se zdroji. 
Alternativní způsob designu API je například Simple Object Access Protocol (SOAP), který dovoluje volání akcí na serveru.
REST API je standard mezi moderními komunikacemi a vzhledem k cíli vytvořit rozšiřitelnou aplikaci je REST nejvhodnější možnost.

Server bude získávat data z URL adresy, těla a hlavičky požadavku.
Požadavky i odpovědi na a ze serveru budou komunikovat zejména v JavaScript Object Notation (JSON).
JSON je jednoduchý formát, který se jednoduše implementuje skoro v každém jazyce a je postaven na bázi seznamu klíčů a příslušných hodnot.
Hodnoty mají mnoho podporovaných datových typů.
Server by měl být však připraven na jakýkoliv formát dat.
Mělo by být možné velmi jednoduše změnit například na formát Extensible Markup Language (XML).

Serverová a klientská část by měly spolu sdílet rozhraní definující vstupní a výstupní data aplikace.
Těmto objektům se říká Data Transfer Object (DTO).
Tyto objekty dovolují jednoduše typovat data, které vstupují a vystupují do a respektive z REST API.

Některé endpointy a zdroje API musí být zabezpečené pomocí autorizace a autentizace. 
O zabezpečení budu mluvit v pozdější kapitole \ref{text:navrh/auth}.
Některé endpointy musí být limitované i rate-limitery, tedy že, omezí počet volání daného endpointu pro konkrétního uživatele či globálně.
Jedná se např. o stránku s registrací.

Samotná serverová část bude tvořena v jazyce TS.
Volba TS je stejná jako v klientské části (viz kapitola \ref{text:navrh/klient}) a to tedy, že nutí programátora psát udržitelný kód, který je staticky typově kontrolován.

Implementace komunikačního uzlu je možná přes řadu způsobů a knihoven.
V JS, resp. TS ekosystému patří mezi nejoblíbenější back-endový rámec knihovna Express.js (často taktéž Express).
Alternativní jako nejoblíbenější bývá\todo{zdroj} zmiňován framework Koa či Fastify.
Právě Express jsem se rozhodl použít díky své malé velikosti, komunitě a jeho velké stopě v ekosystému.
Express má řadu předpřipravených funkcionalit jako jsou routery, middleware a mnoho dalšího.

Server nejspíše bude používat i řadu dalších knihoven.
Taktéž bude muset být obecně zabezpečen, například kvůli Cross-origin Resource Sharing (CORS) a dalším podobným mechanizmům.

\subsubsection{Vrstvy a zodpovědnosti}

I serverová část bude rozdělena do vrstev. 
V realizované aplikaci budou vrstvy pak dále děleny a bude vytvořena další abstrakce.
Níže jsou naplánované veškeré vrstvy pro serverovou část.

% \begin{description}
%     \item[Routery]
%     \item[Middlewares]
%     \item[Kontrolery]
%     \item[Služby]
%     \item[Repozitáře]
%     \item[Modely]
% \end{description}


\subsection{Databáze}\label{text:navrh/databaze}

\section{Obsah}\label{text:navrh/obsah}

\section{Autentizace a autorizace}\label{text:navrh/auth}

\section{Uživatelské prostředí}

\subsection{Komponenty}

