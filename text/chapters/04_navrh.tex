\chapter{Návrh}

% \section{Požadavky a funkcionality}

\section{Architektura}

Celá aplikace bude rozdělena do několika částí, které v této kapitole vysvětlím a popíšu svá rozhodnutí.

\subsection{Klient}

Klientem celé aplikace bude webová stránka, ve které budou uživatelé vytvářet, přehrávat a procházet materiály. 
Obecně na této části budou uživatelé ovládat celou aplikaci.
Webová stránka je vhodná, protože to dovolí rychlejší přístup ke všem funkcionalitám bez nutnosti cokoliv instalovat.
Instalace taktéž není vhodná kvůli velkému počtu používání materiálů studenty, kterým by toto stahování mohlo přijít nevhodné.
Později by šlo z aplikace vytvořit nativní aplikaci na desktop či telefony pomocí enkapsulace webové stránky.

Tato část bude pomocí REST API komunikovat se serverovou částí za pomocí protokolu HTTP, ze které bude získávat veškerá potřebná data.
Restful API je ve zkratce takové API, které pracuje se zdroji jako takovými a dovoluje nad nimi volat klasické akce jako čtení, přidání, smazání a změnu (CRUD).
Mimo jiné definuje další standardizované chování, které by toto API mělo mít.
Tento způsob dovoluje tvořit tzv. \enquote{thick klienty}, které vysvětlím a zdůvodním později.

Klient tedy poběží ve webovém prohlížeči uživatele.
Díky tomu tedy musí být klient definován a stylován za pomocí HTML a CSS.
Kontrola samotné stránky musí být ve skriptovacím jazyce JavaScript.
Pro bezpečnější a udržitelnější kód však v této aplikaci použiji nástavbu TypeScript, která se kompiluje do čistého JavaScriptu.
Klient bude tvořen v myšlence \enquote{thick klienta}, tedy většina funkcionality by měl


\subsection{Server}
\subsection{Databáze}

web? proc?
backend, databaze

\section{Návrh uživatelského prostředí}