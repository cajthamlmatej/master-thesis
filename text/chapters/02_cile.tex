\chapter*{Cíle a metodika práce}
\addcontentsline{toc}{chapter}{Cíle a metodika práce}
\markboth{Cíle a metodika práce}{Cíle a metodika práce}

Tato diplomová práce si klade za cíl návrh a implementaci webové platformy pro tvorbu a sdílení interaktivních výukových materiálů, která umožní pedagogům intuitivní tvorbu obsahu a studentům aktivní zapojení do výuky. 
Klíčovým požadavkem je snadná rozšiřitelnost systému prostřednictvím komunitních rozšíření a podpora různorodých forem vzdělávacího obsahu.

Na počátku práce bude provedena rešerše existujících nástrojů, jejichž účelem je tvorba, správa a distribuce výukových materiálů. 
Cílem této části bude identifikovat jejich silné a slabé stránky, především v~oblasti interaktivity, přizpůsobitelnosti a možnosti spolupráce. 
Výstupem této analýzy budou konkrétní požadavky, které budou reflektovány v~návrhu vlastní platformy. 
Zvláštní důraz by měl být kladen na bezpečnost, udržitelnost a otevřenost systému.

Na základě těchto poznatků bude navržena architektura webové aplikace, jež zohledňuje potřeby různých cílových skupin -- pedagogů, studentů i~vývojářské komunity.
Výsledný návrh bude zahrnovat technickou specifikaci, návrh uživatelského rozhraní, způsob vykreslování obsahu a možnosti integrace rozšíření.

Praktická část práce se bude věnovat implementaci jádra platformy, včetně editoru, přehrávače, systémů pro správu uživatelů a zabezpečení, a to s~důrazem na intuitivní ovládání a podporu kolaborativní práce.
Aplikace bude otestována během výuky za účasti studentů a učitelů.
Tato finální fáze s~uživateli má za úkol prověřit využitelnost celého systému v~reálném prostředí a identifikovat oblasti, které vyžadují další zlepšení.

Struktura práce reflektuje výše popsaný přístup. 
První kapitola je věnována analýze existujících řešení, druhá kapitola návrhu vlastní platformy, třetí kapitola její realizaci, čtvrtá kapitola testování a poslední kapitola diskuzi a zhodnocení. 
Tento postup koresponduje s~metodickým rámcem softwarového inženýrství~\cite{laplante2007software}.

V rešeršní a návrhové části bude zvolen klasický sekvenční model vývoje (tzv.~vodopádový model~\cite{laplante2007software}), který umožňuje systematický sběr požadavků a promyšlený návrh architektury.
Tento přístup je vhodný zejména tehdy, pokud je možné jasně vymezit požadavky ještě před implementací.

Pro samotný vývoj aplikace bude následně aplikován iterativní vývojový model inspirovaný agilními přístupy, konkrétně metodikou typu \textit{evolutionary prototyping}~\cite{somerville2015software}, která umožňuje průběžné zlepšování na základě zpětné vazby od uživatelů. 
V jednotlivých iteracích budou implementovány klíčové funkcionality, ty budou testovány a upravovány podle jejich výsledků.
Z příprav na další iterace vznikne i~seznam nevyřízených věcí, které pak diskutuji v~páté kapitole.

Konkrétní aplikace tohoto vývojového postupu je detailně popsána v~kapitole~\ref{text:realizace/metodikaVyvoje}.
Kombinace předem promyšlené architektury a agilního vývoje zajišťuje jak technickou stabilitu platformy, tak její praktickou použitelnost v~reálném vzdělávacím prostředí.



% \chapter*{Cíle a metodika práce}\addcontentsline{toc}{chapter}{Cíle a metodika práce}\markboth{Cíle a metodika práce}{Cíle a metodika práce}

% Primárním cílem této diplomové práce je vytvoření platformy pro tvorbu a sdílení interaktivních vzdělávacích materiálů, která poskytne pedagogům intuitivní a flexibilní nástroj a studentům umožní efektivní a aktivní formu studia.

% Cílem rešeršní části této diplomové práce je provést analýzu stávajících digitálních nástrojů používaných v~současném vzdělávání a identifikovat jejich silné a slabé stránky.
% Ze zjištěných nedostatků budou sepsány i~požadavky, které by výsledná aplikace měla splňovat.
% Zároveň je nutné zjistit, jak vytvořit bezpečnou a do budoucna udržitelnou platformu na skriptování.
% K~tomu budou analyzovány již existující platformy a bude posouzeno, jaký způsob a technologie jsou nejvhodnější.

% Z~rešerše a analýzy bude navrhnuta funkční platforma, která zohlední potřeby pedagogů, studentů a i~možnosti komunitního rozšíření pomocí skriptování.
% V~návrhu aplikace budou navrhnuty i~použité technologie a zamýšlená architektura celé platformy.
% Výsledné prostředí pro uživatele bude webová aplikace.

% Poté bude tato navržená platforma realizována v~praktické části práce.
% Cílem bude implementovat klíčové funkcionality, které umožní snadnou tvorbu, správu a sdílení interaktivního vzdělávacího obsahu.
% Mimo hlavní části aplikace bude sepsána i~dokumentace a vytvořena hlavní propagační stránka, na které budou prezentovány výhody tohoto systému.
% Výsledná aplikace bude otestována studenty i~pedagogy, tak i~přímo ve výuce z~vytvořených materiálů, k~posouzení využitelnosti aplikace a k~tomu, jaké uživatelské nedostatky obsahuje.

% Poslední částí práce bude posouzení vytvořené platformy a diskuze nad tím, jak by se aplikace mohla změnit a jak bude možné na ni dál navazovat.

% Struktura práce reflektuje toto rozdělení a text je rozdělen do následujících kapitol: \textbf{Analýza}, \textbf{Návrh}, \textbf{Realizace}, \textbf{Testování} a \textbf{Diskuze}.

% Při práci jsem postupoval pomocí dvou metodik.
% Na první sekce -- rešerši a návrh -- jsem používal klasický sekvenční vodopádový vývojový proces, kdy jsem prvně provedl rešerši, poté návrh a pak přešel na další části.

% Na další sekce jsem již využíval iterativní přístup k~vývoji, tedy že jsem inkrementálně pracoval na práci v~cyklech a střídal implementaci, testování a přípravu na další iterace.
% Z~příprav na další iterace vznikl i~seznam nevyřízených věcí, které pak diskutuji v~dané kapitole.
% Konkrétní implementace tohoto iterativního postupu je více popsána v~sekci \ref{text:realizace/metodikaVyvoje}.

% Tento přístup byl zvolen z~důvodu, že stanovení a rešerše klíčových funkcionalit a způsobů byla důležitá ještě před implementací aplikace.
% Agilní způsob byl pro další vývoj vybrán, aby umožnil pružné reakce na zpětnou vazbu a postupné vylepšování navrženého řešení již v~průběhu vývoje. ,
