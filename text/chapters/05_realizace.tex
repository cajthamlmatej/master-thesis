\chapter{Realizace}

\begin{chapterabstract}
Tato kapitola ...
\end{chapterabstract}

\section{Tvorba prototypu}


Během výběru samotného zadání této diplomové práce bylo nutné zvážit řadu faktorů, které by mohly ovlivnit realizovatelnost navrhovaného řešení. 
Jedním z klíčových aspektů byl potenciální problém s editorem, který by měl sloužit jako základ pro interaktivní tvorbu vzdělávacích materiálů. 
Jelikož existovalo riziko, že požadovanou funkcionalitu nebude možné efektivně implementovat v dostupných technologiích, bylo rozhodnuto o vytvoření prototypu. 
Tento prototyp měl za cíl ověřit proveditelnost hlavních funkcí a eliminovat případné zásadní překážky v rané fázi vývoje.

V rámci tvorby prototypu byly implementovány a testovány různé způsoby vykreslování obsahu, přičemž klíčovou otázkou bylo, zda bude vhodnější využít technologii HTML, SVG nebo Canvas.
Každý z těchto přístupů má své specifické výhody a nevýhody, které ovlivňují jak kvalitu zobrazení, tak výkon a možnosti interakce s uživatelem. 
Podrobnější rozbor těchto technologií je uveden v kapitole \ref{text:vykreslovani}, kde jsou jednotlivé přístupy porovnány z hlediska efektivity a použitelnosti v kontextu plánované platformy.

Při vývoji prototypu se objevily i obecné problémy související s implementací editoru -- respektive s jejich matematickou reprezentaci. 
Klíčovou výzvou se ukázaly být skupinové transformace a manipulace s prvky na plátně, zejména pokud jde o jejich pohyb a transformaci v rámci editačního prostoru.
Bylo nutné vyřešit, jakým způsobem budou tyto operace realizovány tak, aby editor zůstal uživatelsky přívětivý, efektivní a zároveň umožňoval pokročilé úpravy vytvořeného obsahu.

Na základě získaných poznatků byl prototyp úspěšně dokončen a posloužil jako důkaz, že zamýšlená funkcionalita může být realizována v rozumném stavu. 
Tento krok umožnil přejít k další fázi vývoje, která se zaměřuje na návrh a implementaci finální verze platformy s důrazem na rozšiřitelnost, komunitní podporu a snadnou použitelnost pro pedagogy i studenty.

Prototyp sloužil především k ověření základního konceptu a nebyl určen pro přímé použití v konečné implementaci platformy. 
Jeho účelem bylo rychlé experimentování s klíčovými funkcionalitami, bez důrazu na optimalizaci kódu či dlouhodobou udržitelnost. 
Po potvrzení, že zamýšlené principy mohou fungovat, bylo nutné přistoupit k přepsání prototypu do robustní a udržitelné podoby. 
Tento krok zahrnoval refaktorování kódu, zpřehlednění architektury a návrh struktury, která umožní snadnou údržbu i budoucí rozšíření funkcionalit. 
Přepis zároveň eliminoval technické dluhy vzniklé při rychlém vývoji.

% - behem vyberu samotneho zadani
% - velky risk problemu s editorem
% - implementovany ruzne zpusoby vykreslovani (viz kapitola \ref{text:vykreslovani}
% - obecny problemy s editorem a matematikou

\section{Metodika vývoje}

Proces vývoje platformy pro tvorbu interaktivních výukových materiálů probíhal iterativním způsobem, což umožnilo průběžné testování a úpravy jednotlivých částí aplikace. 
Tento přístup se ukázal jako efektivní nejen při samotné implementaci, ale i při optimalizaci uživatelského prostředí na základě zpětné vazby od testerů.

Vývoj začal základním návrhem a analýzou požadavků, což bylo detailně popsáno v kapitolách \ref{text:navrh} a \ref{text:analyza}. 
Po definování architektury systému a stanovení klíčových funkcionalit byl zahájen samotný vývoj. 
Každá část aplikace byla pečlivě promyšlena před jejím naprogramováním, přičemž se důraz kladl na modularitu a možnost budoucího rozšíření.

Samotný vývoj probíhal iterativně. 
V každém cyklu bylo nejprve rozhodnuto, na které části aplikace se bude pracovat. 
Poté následovala detailní analýza a návrh této části, což umožnilo identifikovat možné komplikace ještě před samotnou implementací. 
Po naprogramování byla nová funkcionalita nasazena na dočasné testovací prostředí, kde probíhalo její testování různými uživateli. 
K testování byli zapojeni především kolegové a studenti (viz kapitola \ref{text:testovani}), jejichž zpětná vazba byla klíčová pro identifikaci problémů a návrh zlepšení.

Získané podněty byly zaznamenány a klasifikovány dle priority. 
Kritické chyby byly opraveny okamžitě, zatímco drobné úpravy a návrhy na zlepšení byly zařazeny do plánu budoucího vývoje. 
Tento přístup zajišťoval nejen stabilní vývoj systému, ale také jeho postupné vylepšování bez nutnosti výrazných zásahů do již implementovaných částí.

Během vývoje byl také kladen důraz na refaktoring kódu, což umožnilo udržovat čitelnost a efektivitu implementace. 
Starší části aplikace byly pravidelně přepisovány a optimalizovány s cílem zvýšit výkon a usnadnit další rozšíření systému.

Tento iterativní proces vývoje umožnil nejen průběžné zdokonalování aplikace, ale také pružně reagovat na nové požadavky a potřeby uživatelů. 
Díky tomuto přístupu bylo možné zajistit vysokou kvalitu finální verze platformy a její snadnou škálovatelnost pro budoucí rozšíření.


% - po zakladnim navrhu a analyze (viz kapitoly \ref{text:navrh} a \ref{text:analyza})
% - iterativni zpusob prace
%     - rozhodnout co dale programovat, analyza a navrh dane casti
%     - naprogramovat cast
%     - cast nasadit na docasnou produkci
%     - predat k testovani ruznymi kamarady (viz kapitola \ref{text:testovani})
%     - ulozit si feedback, issues
%     - opravit akutni chyby
% - prepisovani a vyuziti refactoring

\section{Serverová část}

\subsection{Autentizace a autorizace}

\section{Klientská část}

\subsection{Designový návrh}

\subsection{Komponenty}

\subsection{Editor a přehrávač}

\subsection{Optimalizace pro webové vyhledávače}

\section{Sestavení a nasazení aplikace}

\section{Dokumentace}

\section{Vytvořené bloky a rozšíření}

\section{Ukázkové materiály}
